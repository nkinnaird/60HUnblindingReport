\chapter{Analysis Procedures}

\section{Key parameters in reconstruction method}

Find out procedures used in 60 hr production dataset

\section{Analysis Data Preparation Procedure}

\begin{itemize}
	\item{git branch: gm2analyses branch feature/KinnairdAnalyses}
	\item{Majority of code located in gm2analyses/macros/RatioMacro folder.}
\end{itemize}

\begin{enumerate}
	\item{Submit jobs to OSG to run the rootTreesAndLostMuons.fcl file which produces root trees of positron hits using the ClusterTree analyzer module and coincident MIP hits using the TestCoincidenceFinder analyzer module.}
	\item{Submit jobs to Fermigrid to produce histograms from root trees using the ClusterTreeToHistsPileup.C macro in RatioMacro/HistMaking. Beyond standard threshold histograms this macro produces pileup and lost muon histograms all within the same root file.}
\end{enumerate}

\section{Histogramming Procedure}

Method: Weighted Ratio (threshold)

\begin{enumerate}
	\item{Loop through all clusters and apply an artificial deadtime to combine hits within 6 ns into a single pulse using the same procedure and code that the pileup method uses (see below). Drop clusters with time $< 25 \mu s$ or time $> 600 \mu s$.}
	\item{Histograms are constructed with ROOT's TH1F class with 149.15 ns bins from $0 - 699.96095 \mu s$ corresponding to 4693 bins.}
	\item{Randomize times by $\pm 149.15/2 ns$ and fill histograms for energies $> 1.7$ GeV. Randomization uses ROOT's default TRandom3 class.}
	\item{Generate a random number per cluster to determine which }
\end{enumerate}

In the code, this amounts to generating a random double between 0 and 1 per pulse and then filling the associated histogram based according to the relative probabilities of the different weights.
 As a reminder, the pulses that had their times shifted by $t \rightarrow t - T/2$ are the pulses that should be weighted by $e^{T/2\tau}$ since they will be filled into the $u_{+}$ histogram, and vice versa for the $u_{-}$ histogram.
