\chapter{Analysis Procedures}


\section{Key parameters in reconstruction method}

Find out procedures used in 60 hr production dataset


\section{Analysis Data Preparation Procedure}

\begin{itemize}
	\item{git branch: gm2analyses branch feature/KinnairdAnalyses}
	\item{Majority of code located under gm2analyses/macros/RatioMacro}
\end{itemize}

\begin{enumerate}
	\item{Submit jobs to OSG to run the rootTreesAndLostMuons.fcl file which produces root trees of positron hits using the ClusterTree analyzer module and coincident MIP hits using the TestCoincidenceFinder analyzer module.}
	\item{Submit jobs to Fermigrid to produce histograms from root trees using the ClusterTreeToHistsPileup.C macro in RatioMacro/HistMaking. Beyond standard threshold histograms this macro produces pileup and lost muon histograms all within the same root file.}
\end{enumerate}


\section{Histogramming Procedure}

Method: Weighted Ratio (threshold)

\begin{enumerate}
	\item{Loop through all clusters and apply an artificial deadtime (ADT) to combine hits within 6 ns into a single pulse using the same procedure and code that the pileup method uses (see below). Drop clusters with time $< 25 \mu s$ or time $> 600 \mu s$.}
	\item{Histograms are constructed with ROOT's TH1F class with 149.15 ns bins from $0 - 699.96095 \mu s$ corresponding to 4693 bins.}
	\item{Randomize times by $\pm 149.15/2$ ns and fill histograms for energies $> 1.7$ GeV. Randomization uses ROOT's default TRandom3 class.}
	\item{Fill one of the four histograms \{$u_{+}(t), u_{-}(t), v_{1}(t), v_{2}(t)$\} as shown in Equation \ref{eqn:fourHists} per cluster. The associated histogram is determined by generating a random number between 0 and 1, and comparing that number to the relative probabilities of the different weights.}
	\item{Clusters filled into the $u_{+}(t)$ histogram have their times shifted by $t \rightarrow t - T/2$ and clusters filled into the $u_{-}(t)$ histogram have their times shifted by $t \rightarrow t + T/2$.}
\end{enumerate}


\section{Gain Correction Procedure}

Gain correction method: Default by the Italian Calibration Team

\begin{enumerate}
	\item{Long term gain is corrected using out-of-fill lasers included normalization from the Source Monitor.}
	\item{In-fill gain is corrected using in-fill lasers including normalization from the Source Monitor.}
	\item{Short-term double pulse (SDTP) effect is not included.}
\end{enumerate}


\section{Pileup Correction Procedure}

Pileup correction method: Asymmetric shadow window

\begin{enumerate}
	\item{Create a vector of clusters per calorimeter per fill. For each cluster look for a second cluster in a window from 12-18 ns after the time of the first cluster. This corresponds to a shadow dead time (SDT) of 6 ns and a shadow gap time (SGT) of 12 ns, equal to 1 and 2 times the applied ADT respectively.}
	\item{Create shadow doublets with energies and times as:
		\begin{gather*}
			E_{doublet} = E_{1} + E_{2} \\
			t_{doublet} = \frac{t_{1}*E_{1} + (t_{2}-SGT)*E_{2}}{E_{1} + E_{2}}
		\end{gather*}
		}
	\item{For each calorimeter construct a pileup spectrum P = doublets - singlets = D - S, where the singlets are subtracted at time $t_{doublet}$, and pulses are only added or subtracted if they are above 1.7 GeV.}


U and V splitting as well
carefule with threhsholds
time randomization of pileup hits

	\item{errors}
	\item{triple pileup not included}
\end{enumerate}










