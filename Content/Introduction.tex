\chapter{Introduction}
\label{Ch:Introduction}

This report details the analysis procedures, results, and systematic studies I carried out for the ``60H'' dataset acquired during Run 1 in 2018 of the Fermilab Muon \gmtwo Experiment, E989. The 60H dataset was gathered from April 22nd, 2018 to April 24th, 2018, with runs 15921 - 15992. The quad voltages used were 13.1 and 18.3 kV, corresponding to an effective n value of 0.10843 (Cite DocDB 11547? Or find a newer reference?). The kicker voltage range was 128 - 132 kV. Nearly $\SI{1e9}{}$ positrons were collected during this period, corresponding to about 10\% of the BNL statistics gathered. (Cite DocDB 14071 by James?)

In Chapter \ref{Ch:Procedures} I detail the analysis procedures used. This includes information on how the data is prepared, starting after the ``production'' stage and leading up to the construction of histograms. The application of gain corrections, pileup construction, the extraction of lost muons, and other various data preparation procedures are described briefly. Also included are various models used in the fitting of the data, including coherent betatron (CBO) effects and the lost muon function.

In Chapter \ref{Ch:Results} I detail the results of the analysis. This includes the application of the pileup correction, the fits to the data and subsequent residuals and FFTs of those residuals, and start time scans. Results are shown both for all calorimeters added together and for individual calorimeters.

In Chapter \ref{Ch:Systematics} I detail the results of systematic studies to the data and fits, which is the real meat of the analysis. This includes studies relating to corrections applied (gain, pileup, etc.), as well as models used within the fit (CBO, lost muons, VW, etc.). Also included are various studies related to items like the bin width or randomization used in the analysis.

Chapter \ref{Ch:Conclusion} simply concludes the final results of the report, as well as next steps for the analysis. There is also an Appendix \ref{Appendix} which provides some derivations for the ratio method used in fitting the data.
